\documentclass{article}

\usepackage{fancyhdr}
\usepackage{extramarks}
\usepackage{amsmath}
\usepackage{amsthm}
\usepackage{amsfonts}
\usepackage{tikz}
\usepackage[plain]{algorithm}
\usepackage{algpseudocode}

\usetikzlibrary{automata,positioning}

%
% Basic Document Settings
%

\topmargin=-0.45in
\evensidemargin=0in
\oddsidemargin=0in
\textwidth=6.5in
\textheight=9.0in
\headsep=0.25in

\linespread{1.1}

\pagestyle{fancy}
\lhead{\hmwkAuthorName}
\chead{\hmwkClass \: \hmwkTitle}
\rhead{\firstxmark}
\lfoot{\lastxmark}
\cfoot{\thepage}

\renewcommand\headrulewidth{0.4pt}
\renewcommand\footrulewidth{0.4pt}

\setlength\parindent{0pt}

%
% Create Problem Sections
%

\newcommand{\enterProblemHeader}[1]{
    \nobreak\extramarks{}{Problem \arabic{#1} continued on next page\ldots}\nobreak{}
    \nobreak\extramarks{Problem \arabic{#1} (continued)}{Problem \arabic{#1} continued on next page\ldots}\nobreak{}
}

\newcommand{\exitProblemHeader}[1]{
    \nobreak\extramarks{Problem \arabic{#1} (continued)}{Problem \arabic{#1} continued on next page\ldots}\nobreak{}
    \stepcounter{#1}
    \nobreak\extramarks{Problem \arabic{#1}}{}\nobreak{}
}

\newcommand{\Fr}{\text{Fr}}
\newcommand{\Char}{\text{Char}}


\setcounter{secnumdepth}{0}
\newcounter{partCounter}
\newcounter{homeworkProblemCounter}
\setcounter{homeworkProblemCounter}{1}
\nobreak\extramarks{Problem \arabic{homeworkProblemCounter}}{}\nobreak{}

%
% Homework Problem Environment
%
% This environment takes an optional argument. When given, it will adjust the
% problem counter. This is useful for when the problems given for your
% assignment aren't sequential. See the last 3 problems of this template for an
% example.
%
\newenvironment{homeworkProblem}[1][-1]{
    \ifnum#1>0
        \setcounter{homeworkProblemCounter}{#1}
    \fi
    \section{Problem \arabic{homeworkProblemCounter}}
    \setcounter{partCounter}{1}
    \enterProblemHeader{homeworkProblemCounter}
}{
    \exitProblemHeader{homeworkProblemCounter}
}

%
% Homework Details
%   - Title
%   - Due date
%   - Class
%   - Section/Time
%   - Instructor
%   - Author
%

\newcommand{\hmwkTitle}{Homework 1}
\newcommand{\hmwkDueDate}{January 15, 2025}
\newcommand{\hmwkClass}{Math 114C}
\newcommand{\hmwkClassInstructor}{Professor Arant}
\newcommand{\hmwkAuthorName}{\textbf{Darsh Verma}}
\newcommand{\raw}{\rightarrow}

%
% Title Page
%

\title{
    \vspace{2in}
    \textmd{\textbf{\hmwkClass:\ \hmwkTitle}}\\
    \normalsize\vspace{0.1in}\small{Due\ on\ \hmwkDueDate\ at 11:59pm}\\
    \vspace{0.1in}\large{\hmwkClassInstructor}
    \vspace{3in}
}

\author{\hmwkAuthorName}
\date{}

\renewcommand{\part}[1]{\textbf{\large Part \Alph{partCounter}}\stepcounter{partCounter}\\}

%
% Various Helper Commands
%

% Useful for algorithms
\newcommand{\alg}[1]{\textsc{\bfseries \footnotesize #1}}

% For derivatives
\newcommand{\deriv}[1]{\frac{\mathrm{d}}{\mathrm{d}x} (#1)}

% For partial derivatives
\newcommand{\pderiv}[2]{\frac{\partial}{\partial #1} (#2)}

% Integral dx
\newcommand{\dx}{\mathrm{d}x}
\newcommand{\R}{\mathbb{R}}
\newcommand{\N}{\mathbb{N}}
\newcommand{\Z}{\mathbb{Z}}

% Alias for the Solution section header
\newcommand{\solution}{\textbf{\large Solution}}

% Probability commands: Expectation, Variance, Covariance, Bias
\newcommand{\E}{\mathrm{E}}
\newcommand{\Var}{\mathrm{Var}}
\newcommand{\Cov}{\mathrm{Cov}}
\newcommand{\Bias}{\mathrm{Bias}}

\begin{document}

\maketitle

\pagebreak

\begin{homeworkProblem} % problem 1
    \textbf{Part (a)} \\
    
    \begin{center}
    \begin{tabular}{c c c c}
        \hline
        $\text{Char}_R(x)$ & $\text{Char}_Q(x)$ & $\text{Char}_{R \& Q}(x)$ & $\text{Char}_R(x) \cdot \text{Char}_Q(x)$ \\
        \hline
        1 & 1 & 1 & $1 \cdot 1 = \mathbf{1}$ \\
        1 & 0 & 0 & $1 \cdot 0 = \mathbf{0}$ \\
        0 & 1 & 0 & $0 \cdot 1 = \mathbf{0}$ \\
        0 & 0 & 0 & $0 \cdot 0 = \mathbf{0}$ \\
        \hline
    \end{tabular}
    \end{center}
    
    \textbf{Part (b)} \\
    \begin{center}
        \begin{tabular}{c c c c}
            \hline
            $\text{Char}_R(x)$ & $\text{Char}_Q(x)$ & $\text{Char}_{R \vee Q}(x)$ & $\min(1, \text{Char}_R(x) + \text{Char}_Q(x))$ \\
            \hline
            1 & 1 & 1 & $\min(1, 2) = 1$ \\
            1 & 0 & 1 & $\min(1, 1) = 1$ \\
            0 & 1 & 1 & $\min(1, 1) = 1$ \\
            0 & 0 & 0 & $\min(1, 0) = 0$ \\
            \hline
        \end{tabular}
    \end{center}
\end{homeworkProblem}
\newpage
\begin{homeworkProblem}
    \textbf{Part (a)}
    \begin{proof}
        First, we prove the forward direction. Assume $A=B$ for subsets $A, B \subseteq \N$. By definition of set equality, $x \in A \iff x \in B$. If $x \in A$, then $x \in B$, and thus $\Char_A(x) = 1$ and $\Char_B(x) = 1$. Similarly, if $x \notin A$, then $x \notin B$, and thus $\Char_A(x) = 0$ and $\Char_B(x) = 0$. \\
        Now, we prove the backward direction. Assume $\Char_A = \Char_B$. We have $x \in A \iff \Char_A(x) = 1 \iff \Char_B(x) = 1 \iff x \in B$ and thus, $A=B$.
    \end{proof}
    \textbf{Part (b)} 
    \begin{proof}
        First, we prove existence. Let $f : \N \raw \{0, 1\}$ be a total function. Let's construct $A$ as follows:
        $$A = \{n \in \N \mid \text{if $f(n) = 1$}\}$$
        By definition, $f = \Char_A$. \\
        Now, we prove uniqueness. Assume for the sake of contradiction that there exists two distinct sets $A$ and $B$ with this property. Then, we have function $f = \Char_A = \Char_B$. By part (a), we have $A = B$, which is a contradiction!
    \end{proof}
    
\end{homeworkProblem}
\newpage
\begin{homeworkProblem} % problem 3
    \textbf{Part (a)} \\
    $f \circ g$ is also the unique unary partial function with empty domain. If $g$ has an empty domain, then $f \circ g$ must also have an empty domain.\\ \\
    \textbf{Part (b)} \\
    $$(f \circ g)(x) = \begin{cases}
        1 & \text{if $x \geq 10$ and $x$ is even} \\
        \uparrow & \text{otherwise}
    \end{cases}$$
    Applying $g$ yields outputs greater than 20 for inputs greater than 10. \\ \\
    \textbf{Part (c)} \\
    $$(f \circ g)(x) = \begin{cases}
        1 & \text{if $x$ is a multiple of 15} \\
        \uparrow & \text{otherwise}
    \end{cases}$$
\end{homeworkProblem}
\newpage
\begin{homeworkProblem} % problem 4
    $$g(x) = 0$$
    $$h(x, t, y) = t + x$$
    \begin{proof}
        We proceed with mathematical induction on $y$. Our base case is when $y = 0$. We get $ M(x, 0) = g(0) = 0$ which is true. Now, we assume $M(x, y) = xy$, and want to show that $M(x, y + 1) = x(y + 1) = xy + x$. By our definition, we have:
        $$M(x, y + 1) = h(x, M(x, y), y) = x + M(x, y) = x + xy = x(y + 1)$$.
    \end{proof}
\end{homeworkProblem}
\newpage
\begin{homeworkProblem}
    $$F(x, y) = g(x)$$
    $$F(x, y + 1) = h(x, F(x, y), y)$$ where
    $$g(x) = x$$
    $$h(x, t, y) = f(t)$$
    We have $F(x, y) \downarrow \iff F(x, z)\downarrow \forall z < y$.
\end{homeworkProblem}
\newpage
\begin{homeworkProblem}
    $f(1) = \mu y[g(1, y) = 0] = \uparrow$. No number bigger than 1 also divides 1. \\
    $f(4) = \mu y[g(4, y) = 0] = 2$. 2 is the smallest non-one factor of 4. \\
    $f(7) = \mu y[g(1, y) = 0] = \uparrow$. 7 is prime. \\
    $f(15) = \mu y[g(1, y) = 0] = 3$. 3 is the smallest non-one factor of 15.
\end{homeworkProblem}
\newpage
\begin{homeworkProblem}
    $$f(x, y) = \begin{cases}
        0 & \text{if $y^2 \geq x$} \\
        1 & \text{if $y^2 < x$}
    \end{cases}$$
    $$g(x) = \mu y [f(x, y) = 0]$$
    If $x$ is not a perfect square, $g$ returns $\lceil \sqrt{x} \rceil$. This function is recursive by definition by minimization. Squaring a natural number (definition by primitive recursion of multiplication)  and the $\geq$ and $<$ relations (defintion by primitve recursion) are recursive.
\end{homeworkProblem}
\newpage
\begin{homeworkProblem}
    $h$ is recrusive by definition by primitive recursion. We have our base case of $h(\vec{x}, 0) = g(\vec{x})$, where $g(\vec{x}) = 1$. We also have $h(\vec{x}, y + 1) = \alpha (\vec{x}, h(\vec{x}, y), y)$, where $\alpha (\vec{x}, t, y) = t \cdot f(\vec{x}, y)$. The function $\alpha$ is recrusive because $f$ is recursive and multiplication is recursive by defnition by primitive recursion.
\end{homeworkProblem}
\end{document}