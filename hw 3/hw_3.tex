\documentclass{article}

\usepackage{fancyhdr}
\usepackage{extramarks}
\usepackage{amsmath}
\usepackage{amsthm}
\usepackage{amsfonts}
\usepackage{amssymb}
\usepackage{tikz}
\usepackage[plain]{algorithm}
\usepackage{algpseudocode}

\usetikzlibrary{automata,positioning}

%
% Basic Document Settings
%

\topmargin=-0.45in
\evensidemargin=0in
\oddsidemargin=0in
\textwidth=6.5in
\textheight=9.0in
\headsep=0.25in

\linespread{1.1}

\pagestyle{fancy}
\lhead{\hmwkAuthorName}
\chead{\hmwkClass \: \hmwkTitle}
\rhead{\firstxmark}
\lfoot{\lastxmark}
\cfoot{\thepage}

\renewcommand\headrulewidth{0.4pt}
\renewcommand\footrulewidth{0.4pt}

\setlength\parindent{0pt}

%
% Create Problem Sections
%

\newcommand{\enterProblemHeader}[1]{
    \nobreak\extramarks{}{Problem \arabic{#1} continued on next page\ldots}\nobreak{}
    \nobreak\extramarks{Problem \arabic{#1} (continued)}{Problem \arabic{#1} continued on next page\ldots}\nobreak{}
}

\newcommand{\exitProblemHeader}[1]{
    \nobreak\extramarks{Problem \arabic{#1} (continued)}{Problem \arabic{#1} continued on next page\ldots}\nobreak{}
    \stepcounter{#1}
    \nobreak\extramarks{Problem \arabic{#1}}{}\nobreak{}
}

\newcommand{\Fr}{\text{Fr}}
\newcommand{\Char}{\text{Char}}


\setcounter{secnumdepth}{0}
\newcounter{partCounter}
\newcounter{homeworkProblemCounter}
\setcounter{homeworkProblemCounter}{1}
\nobreak\extramarks{Problem \arabic{homeworkProblemCounter}}{}\nobreak{}

%
% Homework Problem Environment
%
% This environment takes an optional argument. When given, it will adjust the
% problem counter. This is useful for when the problems given for your
% assignment aren't sequential. See the last 3 problems of this template for an
% example.
%
\newenvironment{homeworkProblem}[1][-1]{
    \ifnum#1>0
        \setcounter{homeworkProblemCounter}{#1}
    \fi
    \section{Problem \arabic{homeworkProblemCounter}}
    \setcounter{partCounter}{1}
    \enterProblemHeader{homeworkProblemCounter}
}{
    \exitProblemHeader{homeworkProblemCounter}
}

%
% Homework Details
%   - Title
%   - Due date
%   - Class
%   - Section/Time
%   - Instructor
%   - Author
%

\newcommand{\hmwkTitle}{Homework 3}
\newcommand{\hmwkDueDate}{February 8, 2026}
\newcommand{\hmwkClass}{Math 114C}
\newcommand{\hmwkClassInstructor}{Professor Arant}
\newcommand{\hmwkAuthorName}{\textbf{Darsh Verma}}
\newcommand{\raw}{\rightarrow}

%
% Title Page
%

\title{
    \vspace{2in}
    \textmd{\textbf{\hmwkClass:\ \hmwkTitle}}\\
    \normalsize\vspace{0.1in}\small{Due\ on\ \hmwkDueDate\ at 11:59pm}\\
    \vspace{0.1in}\large{\hmwkClassInstructor}
    \vspace{3in}
}

\author{\hmwkAuthorName}
\date{}

\renewcommand{\part}[1]{\textbf{\large Part \Alph{partCounter}}\stepcounter{partCounter}\\}

%
% Various Helper Commands
%

% Useful for algorithms
\newcommand{\alg}[1]{\textsc{\bfseries \footnotesize #1}}

% For derivatives
\newcommand{\deriv}[1]{\frac{\mathrm{d}}{\mathrm{d}x} (#1)}

% For partial derivatives
\newcommand{\pderiv}[2]{\frac{\partial}{\partial #1} (#2)}

% Integral dx
\newcommand{\dx}{\mathrm{d}x}
\newcommand{\R}{\mathbb{R}}
\newcommand{\N}{\mathbb{N}}
\newcommand{\Z}{\mathbb{Z}}

% Alias for the Solution section header
\newcommand{\solution}{\textbf{\large Solution}}

% Probability commands: Expectation, Variance, Covariance, Bias
\newcommand{\E}{\mathrm{E}}
\newcommand{\Var}{\mathrm{Var}}
\newcommand{\Cov}{\mathrm{Cov}}
\newcommand{\Bias}{\mathrm{Bias}}


\begin{document}

\maketitle

\pagebreak
    %%%%%%%%%%%%%%%%%%%%%%%%%%%%%%%%%%%%%%%%%%%%%%%%%%%%%%%%%%%%
    %% Problem 1
    %%%%%%%%%%%%%%%%%%%%%%%%%%%%%%%%%%%%%%%%%%%%%%%%%%%%%%%%%%%%
    \begin{homeworkProblem}

    \textbf{Claim.} For every $m, n \ge 1$, there is a recursive total $(m+1)$-ary
    function $s^{m}_{n}(e,\vec{x})$ such that
    \[
        \varphi^{(m+n)}_{e}(\vec{x},\vec{y})
        \cong
        \varphi^{(n)}_{s^{m}_{n}(e,\vec{x})}(\vec{y})
    \]
    for all $\vec{x}\in\N^{m}$ and $\vec{y}\in\N^{n}$.

    \begin{proof}
    We proceed by induction on $m$.

    \textbf{Base case} ($m=1$).
    This is exactly the s-m-n theorem proved in class (with parameter $n$):
    there exists a recursive total function $s^{1}_{n}(e,x)$ such that
    \[
        \varphi^{(1+n)}_{e}(x,\vec{y})
        \cong
        \varphi^{(n)}_{s^{1}_{n}(e,x)}(\vec{y}).
    \]

    \textbf{Inductive step.}
    Suppose the result holds for $m$; we prove it for $m+1$.
    Let $\vec{x}=(x_1,\dots,x_{m+1})\in\N^{m+1}$ and $\vec{y}\in\N^{n}$.
    Write
    \[
        \varphi^{((m+1)+n)}_{e}(x_1,x_2,\dots,x_{m+1},\vec{y})
        =
        \varphi^{(1+(m+n))}_{e}(x_1,\;x_2,\dots,x_{m+1},\vec{y}).
    \]
    By the base case (applied with $n$ replaced by $m+n$), there is a
    recursive total function $s^{1}_{m+n}$ such that
    \[
        \varphi^{(1+(m+n))}_{e}(x_1,\;x_2,\dots,x_{m+1},\vec{y})
        \cong
        \varphi^{(m+n)}_{s^{1}_{m+n}(e,x_1)}(x_2,\dots,x_{m+1},\vec{y}).
    \]
    By the inductive hypothesis (applied to the index $s^{1}_{m+n}(e,x_1)$),
    there is a recursive total function $s^{m}_{n}$ such that
    \[
        \varphi^{(m+n)}_{s^{1}_{m+n}(e,x_1)}(x_2,\dots,x_{m+1},\vec{y})
        \cong
        \varphi^{(n)}_{s^{m}_{n}(s^{1}_{m+n}(e,x_1),\,x_2,\dots,x_{m+1})}(\vec{y}).
    \]
    Define
    \[
        s^{m+1}_{n}(e,x_1,\dots,x_{m+1})
        \;=\;
        s^{m}_{n}\!\bigl(s^{1}_{m+n}(e,x_1),\;x_2,\dots,x_{m+1}\bigr).
    \]
    This is a composition of recursive total functions, so it is recusive and
    total.  Combining the chain of Kleene equalities gives
    \[
        \varphi^{((m+1)+n)}_{e}(\vec{x},\vec{y})
        \cong
        \varphi^{(n)}_{s^{m+1}_{n}(e,\vec{x})}(\vec{y}),
    \]
    completing the induction.
    \end{proof}

    \end{homeworkProblem}

    \newpage

    %%%%%%%%%%%%%%%%%%%%%%%%%%%%%%%%%%%%%%%%%%%%%%%%%%%%%%%%%%%%
    %% Problem 2
    %%%%%%%%%%%%%%%%%%%%%%%%%%%%%%%%%%%%%%%%%%%%%%%%%%%%%%%%%%%%
    \begin{homeworkProblem}

    \textbf{Claim (Rogers' Fixed Point Theorem).}
    If $f:\N\to\N$ is a total recursive function, then for any $n\ge 1$
    there exists $e^{*}\in\N$ such that
    $\varphi^{(n)}_{e^{*}} = \varphi^{(n)}_{f(e^{*})}$.

    \begin{proof}
    Let $f:\N\to\N$ be total recursive and $n\ge 1$.
    Define a partial function $g:\subseteq\N^{n+1}\to\N$ by
    \[
        g(e,\vec{y}) \;=\; \varphi^{(n)}_{f(e)}(\vec{y}).
    \]
    Since $f$ is total recursive and the universal partial recursive function
    $(e,\vec{y})\mapsto\varphi^{(n)}_{e}(\vec{y})$ is partial recursive,
    the composition $g$ is partial recursive (as a fucntion of $n+1$ variables).

    By Kleene's second recursion theorem, there exists $e^{*}\in\N$ such that
    \[
        \varphi^{(n)}_{e^{*}}(\vec{y})
        \;\cong\;
        g(e^{*},\vec{y})
        \;=\;
        \varphi^{(n)}_{f(e^{*})}(\vec{y})
        \qquad\text{for all } \vec{y}\in\N^{n}.
    \]
    Therefore $\varphi^{(n)}_{e^{*}} = \varphi^{(n)}_{f(e^{*})}$.
    \end{proof}

    \end{homeworkProblem}

    \newpage

    %%%%%%%%%%%%%%%%%%%%%%%%%%%%%%%%%%%%%%%%%%%%%%%%%%%%%%%%%%%%
    %% Problem 3
    %%%%%%%%%%%%%%%%%%%%%%%%%%%%%%%%%%%%%%%%%%%%%%%%%%%%%%%%%%%%
    \begin{homeworkProblem}

    \textbf{Claim (Kleene's Second Recursion Theorem).}
    For every $n\ge 1$ and every partial recursive function
    $\Phi:\subseteq\N^{n+1}\to\N$, there exists $e^{*}\in\N$ such that
    \[
        \varphi^{(n)}_{e^{*}}(\vec{y})
        \;\cong\;
        \Phi(e^{*},\vec{y})
        \qquad\text{for all }\vec{y}\in\N^{n}.
    \]

    \begin{proof}
    Let $\Phi:\subseteq\N^{n+1}\to\N$ be partial recursive.
    Since $\Phi$ is partial recursive, it has an index: there exists $a\in\N$
    with $\varphi^{(n+1)}_{a} = \Phi$, i.e.\
    $\varphi^{(n+1)}_{a}(e,\vec{y}) \cong \Phi(e,\vec{y})$ for all
    $e\in\N$, $\vec{y}\in\N^{n}$.

    By the s-m-n theorem (with $m=1$), there is a recursive total function
    $s:\N^{2}\to\N$ such that
    \[
        \varphi^{(n)}_{s(a,e)}(\vec{y})
        \;\cong\;
        \varphi^{(n+1)}_{a}(e,\vec{y})
        \;=\;
        \Phi(e,\vec{y})
        \qquad\text{for all } e,\;\vec{y}\in\N^{n}.
    \]
    Define $h:\N\to\N$ by $h(e) = s(a,e)$.  Since $s$ is recursive total
    and $a$ is a fixed constant, $h$ is a total recursive function.

    By Rogers' fixed point theorem, there exists $e^{*}\in\N$ such that
    \[
        \varphi^{(n)}_{e^{*}} \;=\; \varphi^{(n)}_{h(e^{*})}
        \;=\; \varphi^{(n)}_{s(a,e^{*})}.
    \]
    Therefore, for all $\vec{y}\in\N^{n}$,
    \[
        \varphi^{(n)}_{e^{*}}(\vec{y})
        \;\cong\;
        \varphi^{(n)}_{s(a,e^{*})}(\vec{y})
        \;\cong\;
        \Phi(e^{*},\vec{y}),
    \]
    which is exactly the conclusion of Kleene's second recurison theorem.
    \end{proof}

    \end{homeworkProblem}

    \newpage

    %%%%%%%%%%%%%%%%%%%%%%%%%%%%%%%%%%%%%%%%%%%%%%%%%%%%%%%%%%%%
    %% Problem 4
    %%%%%%%%%%%%%%%%%%%%%%%%%%%%%%%%%%%%%%%%%%%%%%%%%%%%%%%%%%%%
    \begin{homeworkProblem}

    Let $W_e = \operatorname{dom}(\varphi^{(1)}_e)$ for every $e\in\N$.

    \bigskip
    \textbf{(a)} We show there is a recursive total function $u:\N\to\N$ with
    $W_{u(x)} = \{y\in\N : x\mid y\}$.

    \begin{proof}
    Define $g:\N^{2}\to\N$ by
    \[
        g(x,y) =
        \begin{cases}
            0 & \text{if } x\mid y,\\
            \uparrow & \text{if } x\nmid y,
        \end{cases}
    \]
    where $\uparrow$ means ``undefined.''
    Concretely, $g(x,y) = 0\cdot\mu z\,[\text{Div}(x,y)=0]$, where
    $\text{Div}(x,y)$ is defined to equal $0$ if $x\mid y$ (i.e.\ if
    $(\exists k\le y)\,[k\cdot x = y]$) and to go into an infinite search
    otherwise.  More precisely, the relation ``$x$ divides $y$'' is primitive
    recursive (decidable by bounded search: check whether any $k$ with
    $0\le k\le y$ satisfies $k\cdot x = y$), so we can define
    \[
        g(x,y) =
        \begin{cases}
            0 & \text{if } (\exists k\le y)\,[k\cdot x = y],\\
            \uparrow & \text{otherwise},
        \end{cases}
    \]
    which is partial recursive.  (If $x\mid y$, the bounded search succeeds and
    we output 0; otherwise we invoke $\mu z\,[z\ne z]$, which diverges.)

    Since $g$ is partial recursive, there is an index $a$ with
    $\varphi^{(2)}_{a}(x,y) \cong g(x,y)$.
    By the s-m-n theorem, there is a recursive total function $u:\N\to\N$ with
    \[
        \varphi^{(1)}_{u(x)}(y)
        \;\cong\;
        \varphi^{(2)}_{a}(x,y)
        \;=\;
        g(x,y).
    \]
    Then
    \[
        W_{u(x)}
        = \operatorname{dom}(\varphi^{(1)}_{u(x)})
        = \{y\in\N : g(x,y)\!\downarrow\}
        = \{y\in\N : x\mid y\}.  \qedhere
    \]
    \end{proof}

    \bigskip
    \textbf{(b)} We show there exists $e^{*}\in\N$ with
    $W_{e^{*}} = \{y\in\N : e^{*}\mid y\}$.

    \begin{proof}
    By part (a), the function $u:\N\to\N$ is total recursive and satisfies
    $W_{u(x)} = \{y : x\mid y\}$ for all $x$.
    By Rogers' fixed point theorem, there exists $e^{*}\in\N$ such that
    \[
        \varphi^{(1)}_{e^{*}} = \varphi^{(1)}_{u(e^{*})}.
    \]
    Therefore
    \[
        W_{e^{*}}
        = \operatorname{dom}(\varphi^{(1)}_{e^{*}})
        = \operatorname{dom}(\varphi^{(1)}_{u(e^{*})})
        = W_{u(e^{*})}
        = \{y\in\N : e^{*}\mid y\}.  \qedhere
    \]
    \end{proof}

    \end{homeworkProblem}

    \newpage

    %%%%%%%%%%%%%%%%%%%%%%%%%%%%%%%%%%%%%%%%%%%%%%%%%%%%%%%%%%%%
    %% Problem 5
    %%%%%%%%%%%%%%%%%%%%%%%%%%%%%%%%%%%%%%%%%%%%%%%%%%%%%%%%%%%%
    \begin{homeworkProblem}

    For any $e\in\N$, let
    $R_e = \{y\in\N : (\exists x)\;\varphi^{(1)}_e(x)\!\downarrow = y\}$
    denote the range of $\varphi^{(1)}_e$.

    \bigskip
    \textbf{(a)} We show there is a computable total function $S:\N\to\N$ with
    $R_{S(x)} = \{y\in\N : y\text{ is a power of }x\}$.

    \begin{proof}
    Define $f:\N^{2}\to\N$ by $f(x,k) = x^{k}$.  This is a total function,
    and it is primitive recursive (exponentiation is primitive recursive).

    Since $f$ is (total) recursive, there exists an index $a$ with
    $\varphi^{(2)}_{a}(x,k) = f(x,k) = x^{k}$ for all $x,k$.
    By the s-m-n theorem, there is a recursive total function $S:\N\to\N$ with
    \[
        \varphi^{(1)}_{S(x)}(k)
        \;\cong\;
        \varphi^{(2)}_{a}(x,k)
        \;=\;
        x^{k}
        \qquad\text{for all } k\in\N.
    \]
    In particular $\varphi^{(1)}_{S(x)}$ is total and its range is
    \[
        R_{S(x)}
        = \{x^{k} : k\in\N\}
        = \{y\in\N : y\text{ is a power of }x\}.  \qedhere
    \]
    \end{proof}

    \bigskip
    \textbf{(b)} We show there exists $x^{*}\in\N$ with
    $R_{x^{*}} = \{y\in\N : y\text{ is a power of }x^{*}\}$.

    \begin{proof}
    By part (a), $S:\N\to\N$ is total recursive with
    $R_{S(x)} = \{y : y\text{ is a power of }x\}$ for all $x$.
    By Rogers' fixed point theorem, there exists $x^{*}\in\N$ such that
    \[
        \varphi^{(1)}_{x^{*}} = \varphi^{(1)}_{S(x^{*})}.
    \]
    Therefore
    \[
        R_{x^{*}}
        = \operatorname{range}(\varphi^{(1)}_{x^{*}})
        = \operatorname{range}(\varphi^{(1)}_{S(x^{*})})
        = R_{S(x^{*})}
        = \{y\in\N : y\text{ is a power of }x^{*}\}.  \qedhere
    \]
    \end{proof}

    \end{homeworkProblem}

\end{document}